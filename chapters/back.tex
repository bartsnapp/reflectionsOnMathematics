\addcontentsline{toc}{chapter}{Answers to Selected Problems}

\chapter*{Answers to Selected Problems}

\section*{Section 1.1}

\noindent 11. yes

\noindent 12.  yes

\noindent 14.  yes

\noindent 18.  (1, -9)

\noindent 23.  (4, 3)

\noindent 28.  $\left ( \frac{3-2\sqrt{3}}{2}, \frac{3\sqrt{3} + 2}{2} \right )$

\noindent 32.  (0,0)

\noindent 35.  $(\sqrt{3}, 1)$

\noindent 41.  $F_{y=0}$

\noindent 47.   $ \left [  \begin{matrix} 
      1 & 0 & 1 \\
      0 & 1 & 1 \\
      0 & 0 & 1 \\
   \end{matrix} \right ]$


\section*{Section 1.2}

\noindent 2. yes

\noindent 10.  (-5, 0)

\noindent 14.  (-2, 5)

\noindent 15.  (7,0)

\noindent 18.  (-7, 5)

\noindent 20.  $R_{90} T_{(3, -2)}$

\noindent 25.  $T_{(2,3)} R_{45} T_{(-2, -3)}$


\section*{Section 1.3}

\noindent 1.  In your own words

\noindent 4.  $x = 0$, $y = x$, $y = -x$, $y = 0$

\noindent 14.  Many answers.  For example, $F \circ FR$ or $FR \circ FR^2$

\noindent 22.  \vskip 0.1in
\tiny
\noindent \begin{tabular}{|c||c|c|c|c|c|c|c|c|c|c|c|c|}
		\hline
		$\circ$ & $I$ & $R$ & $R^2$ & $R^3$ & $R^4$ & $R^5$ & $F$ & $FR$ & $FR^2$ & $FR^3$ & $FR^4$ & $FR^5$ \\ \hline \hline 
		$I$ & $I$ & $R$ & $R^2$ & $R^3$ & $R^4$ & $R^5$ & $F$ & $FR$ & $FR^2$ & $FR^3$ & $FR^4$ & $FR^5$ \\ \hline
		$R$ & $R$ & $R^2$ & $R^3$ & $R^4$ & $R^5$ & $I$ & $FR^5$ & $F$ & $FR$ & $FR^2$ & $FR^3$ & $FR^4$\\ \hline
		$R^2$ & $R^2$ & $R^3$ & $R^4$ & $R^5$ & $I$ & $R$ & $FR^4$ &  $FR^5$ & $F$ & $FR$ & $FR^2$ & $FR^3$ \\ \hline
		$R^3$ &  $R^3$ & $R^4$ & $R^5$ & $I$ & $R$ & $R^2$ & $FR^3$ & $FR^4$ &  $FR^5$ & $F$ & $FR$ & $FR^2$ \\ \hline
		$R^4$ &  $R^4$ & $R^5$ & $I$ & $R$ & $R^2$ & $R^3$ & $FR^2$ & $FR^3$ & $FR^4$ &  $FR^5$ & $F$ & $FR$ \\ \hline
		$R^5$ & $R^5$ & $I$ & $R$ & $R^2$ & $R^3$ & $R^4$ & $FR$ & $FR^2$ & $FR^3$ & $FR^4$ &  $FR^5$ & $F$  \\ \hline
		$F$ & $F$ & $FR$ & $FR^2$ & $FR^3$ & $FR^4$ & $FR^5$ & $R^3$ & $R^2$ & $R$ & $I$ & $R^5$ & $R^4$ \\ \hline
		$FR$ & $FR$ & $FR^2$ & $FR^3$ & $FR^4$ & $FR^5$ & $F$ & $R^2$ & $R$ & $I$ & $R^5$ & $R^4$ & $R^3$ \\ \hline
		$FR^2$ & $FR^2$ & $FR^3$ & $FR^4$ & $FR^5$ & $F$ & $FR$ & $R$ & $I$ & $R^5$ & $R^4$ & $R^3$ & $R^2$  \\ \hline
		$FR^3$ & $FR^3$ & $FR^4$ & $FR^5$ & $F$ & $FR$ & $FR^2$ & $I$ & $R^5$ & $R^4$ & $R^3$ & $R^2$ & $R$  \\ \hline
		$FR^4$ & $FR^4$ & $FR^5$ & $F$ & $FR$ & $FR^2$ & $FR^3$ & $R^5$ & $R^4$ & $R^3$ & $R^2$ & $R$ & $I$ \\ \hline
		$FR^5$ & $FR^5$ & $F$ & $FR$ & $FR^2$ & $FR^3$ & $FR^4$ &  $R^4$ & $R^3$ & $R^2$ & $R$ & $I$ & $R^5$\\ \hline
		\hline
	\end{tabular}
\normalsize

\section*{Section 2.1}


 \noindent 5.  Step 1: Construct a perpendicular passing through the point. 
 
Step 2: Construct a perpendicular to the new line and also passing through the point.

Step 3: The second perpendicular is your parallel line.


 \noindent 7.  See the notes from class.


\noindent 12.  Step 1: Given a segment $AB$, continue the line. 

Step 2: Construct a perpendicular through $A$.

Step 3: Fold the paper so that the crease is on $A$.  Mark through $B$ onto the line, call this point $C$. 

Step 4: Fold along the line through $B$ and $C$.  Mark through $A$ onto the paper, call this point $D$. 

Step 5.  Draw segment $CD$ and segment $BD$.


\noindent 14.  First show that if the angles are trisected, the resulting triangle is equilateral. 

 - Use folds to show that the triangle is isoceles.  This process should work for all sides of the triangle, so it must be equilateral.
 
 Next show that if the triangle is equilateral, the angles are trisected. 
 
  - Fold the angles on themselves to ``measure''.
  

\noindent 16.  We construct a 3-4-5 triangle.  

Step 1: Construct a segment of length 3.  Call this $AB$. 

Step 2: Construct a perpendicular to line $AB$ through $A$. 

Step 3: Using the segment of length 1, measure 4 units on this perpendicular from $A$.  Call this point $B$ 

Step 4: Draw segment $BC$.


\noindent 17.  See problem \#16, construct a right triangle whose legs are length 1 and 2; the hypotenuse will have length $\sqrt{5}$.


\section*{Section 2.2}


\noindent 4. Both make right angles with a side.  The bisector goes through the midpoint of the side, the altitude connects the side to the opposite vertex.

\noindent 5.  The bisector cuts an angle in two, the median cuts a side in two.

\noindent 6.  Incenter

\noindent 7.  Circumcenter

\noindent 12.  No

\noindent 20.  The median of the hypotenuse

\noindent 24.  Draw a triangle through the vertices.  Find the perpendicular bisector of each side.  The intersection of these points is the circumcenter.

\noindent 27.  Step 1: Label the vertex of the angle $A$.  Name the other end of the segment $B$. 

Step 2: Find the point which is the intersection of the altitude and the side of the angle not containing $B$.  Call this point $C$. 

Step 3: Draw segment $BC$.


\noindent 28.  Step 1: Label the vertex $A$. 

Step 2: Find the intersections of the altitudes with the angles, label these points $B$ and $C$. 

Step 3: Draw $\triangle ABC$.


\section*{Section 2.3}

\noindent 2.  All equilateral triangles have all angles equal to $60^\circ$.

\noindent 4.  The angles sum to $180^\circ$.  If the original triangle has other angles $a$ and $b$, each smaller triangle also has angles $90^\circ$, $a$, and $b$.  (Draw a figure to help you understand this situation.)

\noindent 7.  (a) $b = 1 \quad c = \sqrt{10} \quad f = \sqrt{250}$ \\
(b) $a = 2.4 \quad c = \sqrt{14.76} \quad f = \sqrt{277.16}$ \\
(c) $c = \sqrt{65} \quad d = 12.25 \quad f = \sqrt{491.5625}$ \\
(d) $c = \sqrt{29} \quad e = 4.4 \quad f = \sqrt{140.36}$

\noindent 10.  1451.61m or 4762.5 feet

\noindent 11.  3.095 feet or $37 \frac17$ inches

\noindent 12.  0.012 times as close as the theater.  For instance, if you sat 25 feet from the screen at the theater, you would have to sit around 3.6 inches from the TV at home.

\noindent 15.  230m tall

\noindent 17.  The two large triangles are congruent by SAS (Side-Angle-Side), so the hypotenuse lengths must be equal.


\section*{Section 3.1}

\noindent 3.  There are many examples

\noindent 5.  There are many examples

\noindent 8.  $360^\circ$

\noindent 10.

 \begin{tabular}{c|c|c|c}
	3 & yes & $60^\circ$ & 6 \\
	4 & yes & $90^\circ$ & 4\\
	5& no & $108^\circ$ & \\
	6 & yes & $120^\circ$ & 3 \\
	7 & no &$ 128.6^\circ$ & \\
	8 & no & $135^\circ$ & \\
	9 & no &$ 140^\circ$ & \\
	10 & no & $144^\circ$ & \\
\end{tabular}

(a) The interior angles divide 360 evenly.

(b) As he number of sides increases, the measure of the angle gets closer to $180^\circ$.

(c) A 3-gon, 4-gon, 6-gon.  Perhaps the hexagon is more stable?

\section*{Section 3.2}

\noindent 4. (b), (c), (d), (e)

\noindent 6.  The faces are regular, and the same number meet at each vertex.

\noindent 7. (a) true

(b) false

(c) true

(d)  true

(e) true

\noindent 10.  When excluding a number of angles which would sum to more than $360^\circ$.

\noindent 13.  24

\noindent 18.  $\frac{1}{\sqrt{3}}$

\section*{Section 3.3}

\noindent 1. (a) 1-dimensional 

(b) 1-dimensional 

(c) 2-dimensional 

(d) 2-dimensional (could argue otherwise)

(e) 3-dimensional

(f) 3-dimensional

\noindent 2.  Four points which change into four circles.

\noindent 7.  Draw a tetrahedron by drawing two triangles who share an edge, and connecting corresponding vertices.  Draw a hypertetrahedron by drawing two tetrahedra who share a face, then connecting corresponding vertices.

\noindent 8.  (a) a 2-sphere is a circle because all points are some distance $r$ in 2 dimensions from the center. 

(b) a 3-sphere is a sphere because all points are some distance $r$ in 3 dimensions from the center.

A 1-sphere is two points, and a four sphere is all points in four-dimensional space which are a fixed distance from some centerpoint.  You could describe a 4-sphere centered at the origin of radius 1 as all points $(x, y, z, w)$ who satisfy $\sqrt{x^2 + y^2 + z^2 + w^2} = 1$.

\noindent 10.  (a) perimeter = 24 in \quad area = 24 in$^2$ \quad volume = 8 in$^3$

(b) perimeter = 72 in \quad area = 216 in$^2$ \quad volume = 216 in$^3$

\noindent 12.  (a) 2 units 

(b) 3 units 

(c) 4 units 

(d) 5 units 

(e) $n$ units

\section*{Section 4.1}

\noindent 4.  $\{ \frac41\}$ and $\{ \frac43\}$ look the same.  $\{\frac41\}$ is one piece, $\{\frac42\}$ is two pieces.

\noindent 8.  $\{ \frac81\}$ and $\{ \frac87\}$ look the same, and are one piece.  $\{ \frac82\}$ and $\{ \frac86\}$ look the same, and are two pieces.  $\{ \frac83\}$ and $\{ \frac85\}$ look the same, and are one piece.  $\{ \frac84\}$ is four pieces.

\noindent 9.  $\{ \frac91\}$ and $\{ \frac98\}$ look the same, and are one piece.  $\{ \frac92\}$ and $\{ \frac97\}$ look the same, and are one piece.  $\{ \frac93\}$ and $\{ \frac96\}$ look the same, and are three pieces.  $\{ \frac94\}$ and $\{ \frac95\}$ look the same, and are one piece.

\noindent 13.  $\{ \frac84\}$, $\{ \frac{12}{4}\}$, $\{ \frac{36}{4}\}$, for example.

\noindent 14.  $\{ \frac{10}{5}\}$, $\{ \frac{20}{5}\}$, $\{ \frac{30}{5}\}$, for example.

\noindent 15.  We trace the last two lines two times.  If $s \geq n$ we will just redraw lines we have already drawn.

\noindent 16.  We actually draw the star backwards, so while they look the same, the actual process of drawing the star is different.

\noindent 22.  When we reduce to lowest terms, we reduce the star to just one piece.  Each of the pieces in the star not in lowest terms is the same as this one piece, just in a different position.  Also, the star in lowest terms is the first piece drawn of the star not in lowest terms.

\noindent 23.  You can draw asterisk stars when $n$ is even.  You cannot draw asterisk stars when $n$ is odd.







\addcontentsline{toc}{chapter}{References and Further Reading}
\begin{thebibliography}{99}

\bibitem{abbott} E.A.\ Abbott. \emph{Flatland: A Romance of Many Dimensions}. Penguin Classics, 1998.

%\bibitem{BarkerHowe} W.\ Barker and R.\ Howe. \emph{Continuous Symmetry: From Euclid to Klein}. American Mathematical Society, 2007.

\bibitem{brown} R.G.\ Brown. \emph{Transformational Geometry}. Ginn and Company, 1973.

%\bibitem{conway} J.H.\ Conway and R.K.\ Guy. \emph{The Book of Numbers}. Springer, 1995.

\bibitem{coxeter} H.S.M.\ Coxeter. \emph{Introduction to Geometry}. Wiley Classics Library, 1989.

\bibitem{jeger} M.\ Jeger. \emph{Transformation Geometry}. John Wiley\&Sons, 1966.

\bibitem{macgillavry} C.H.\ MacGillavry. \emph{Fantasy\&Symmetry: The Periodic Drawings of M.C.\ Escher.} Harry N.\ Abrams Inc, 1965.

\bibitem{mason} J.\ Mason with L.\ Burton and K.\ Stacey. \emph{Thinking Mathematically---Revised Edition}. Prentice Hall, 1985.

\bibitem{rosen} J.\ Rosen. \emph{Symmetry in Science.} Springer, 1995.

\bibitem{row} T.S.\ Row. \emph{Geometric Exercises in Paper Folding.} The Open Court Publishing Company, 1901.

\bibitem{sagan} C.\ Sagan. \emph{Cosmos.} Random House, 2002.

\end{thebibliography}


\addcontentsline{toc}{chapter}{Index}
\printindex
