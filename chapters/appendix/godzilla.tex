\newpage
\section{Whose Intimidating Who Now?}

In this activity, we'll investigate problems related to length, area,
and volume. To start, let me tell you a bit about myself. I'm about
$6'$ tall and I weigh about 160 pounds.

\begin{prob}
Imagine if you will, that one day I ``divide by zero'' and I am
fantastically made 100 times taller. Let's call this bigger
me \textit{monster-me.}
\begin{enumerate}
\item How tall is monster-me?
\item Relative to my eyeball, how much surface area does monster-me's eyeball have?
\item How much does monster-me weigh?
\end{enumerate}
\end{prob}


\begin{prob}
While you are probably imagining me as some sort of ``car-eating
monster,'' the reality is much different. I claim that monster-me
would have a hard time moving.  Can you explain why this is true?
\end{prob}

\begin{prob}
I also claim that monster-me would suffocate.  Can you explain why
this is true? Hint: How do lungs work? How does oxygen get to the
brain? How much blood would the monster-me have?
\end{prob}



\begin{prob}
Now imagine if you will, that one day I ``divide by infinity'' and I
am fantastically made 100 times smaller. Let's call this smaller me
\textit{mini-me.}
\begin{enumerate}
\item How tall is mini-me?
\item Relative to my eyeball, about how much surface area does mini-me's eyeball have?
\item How much does mini-me weigh?
\end{enumerate}
\end{prob}


\begin{prob}
I claim that mini-me would be able to fall from a great (relative)
height and be just fine.  Can you explain why this is true?
\end{prob}



\begin{prob}
When I was little, I used to want to fold a giant paper airplane and
actually fly around in it. Would this work? Why or why not?
\end{prob}



