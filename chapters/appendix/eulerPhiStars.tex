\newpage
\section{Pieces of Star Stuff}

In this activity, we'll investigate how many stars $\star{n}{s}$ can
be drawn without lifting your pencil---we'll consider $\star{3}{1}$
and $\star{3}{2}$ to be different stars. Call this number $\phi(n)$.

\begin{prob}
Compute $\phi(2)$, $\phi(3)$, $\phi(4)$, $\phi(5)$, $\phi(6)$,
$\phi(7)$, and $\phi(8)$. 
\end{prob}

\begin{prob}
Given any value of $n$, can you give a simple description of how to
compute $\phi(n)$ \textit{without} drawing any stars?
\end{prob}

\begin{prob}
Use your description to compute $\phi(11)$, $\phi(101)$, $\phi(5051)$.
\end{prob}

\begin{prob}
If $p$ is prime, what is $\phi(p)$? Explain your reasoning.
\end{prob}

\begin{prob}
Compute $\phi(8)$, $\phi(16)$, $\phi(32)$, and $\phi(64)$.
\end{prob}

\begin{prob}
Compute $\phi(9)$, $\phi(27)$, $\phi(81)$, and $\phi(243)$.
\end{prob}

\begin{prob}
Compute $\phi(5)$, $\phi(25)$, $\phi(125)$, and $\phi(625)$.
\end{prob}

\begin{prob}
If $p$ is prime, what is $\phi(p^n)$? Explain your reasoning.
\end{prob}

\begin{prob}
Make a chart of values for $\phi(n)$ for $n = 1,\dots, 20$. Can you
make a conjecture as to how $\phi(a)$ and $\phi(b)$ relate to
$\phi(ab)$?
\end{prob}
