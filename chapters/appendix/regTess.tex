\newpage
\section{How Many Regular Tessellations?}

In this activity we are going to investigate regular tessellations.

\begin{prob}
What is a regular polygon? How many different regular polygons are
there?
\end{prob}


\begin{dfn}\index{convex} 
An object is \textbf{convex} if given any two points inside the object,
the segment connecting those two points is also contained inside the
object.
\end{dfn}


\begin{prob}
Can you explain why a regular polygon must be convex? 
\end{prob}



\begin{dfn}\index{tessellation!regular}\index{regular!tessellation}
A tessellation is called a \textbf{regular tessellation} if it is composed of copies of a single regular polygon and these polygons meet vertex to
vertex.\index{regular!polygon}
\end{dfn}

\begin{prob}
Give an example of a tessellation that is composed of two different copies
of regular polygons.
\end{prob}

\begin{prob}
Give an example of a tessellation that is composed of copies of a
single regular polygon where these polygons don't always meet vertex
to vertex.
\end{prob}

