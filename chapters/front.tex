
\pagenumbering{gobble}


\title{
\textbf{\textsf{
\Huge Reflections on Mathematics \\
\Large Math 1118: Spring 2018
}}}
\author{}
\date{}
\maketitle



\begin{abstract}
Copyright \copyright~2012 Bart Snapp and Jenny George.

\vspace{.5cm}

\noindent
This work is licensed under the Creative Commons:
\begin{center}
Attribution-NonCommercial-ShareAlike License 
\end{center}
To view a copy of this license, visit
\[
\texttt{http://creativecommons.org/licenses/by-nc-sa/3.0/}
\]

\vspace{.5cm}

\noindent This document was typeset on \today.

\end{abstract}



\chapter*{Preface}


These notes are designed with architecture and design students in
mind.  It is our hope that the reader will find these notes both
interesting and challenging.  In particular, there is an emphasis on
the ability to communicate mathematical ideas.  Three goals of these
notes are:
\begin{itemize}
\item To enrich the reader's understanding of geometry and algebra. 
We hope to show the reader that geometry and algebra are deeply
connected.
\item To place an emphasis on problem solving. The reader will be exposed 
to problems that ``fight-back.'' Worthy minds such as yours deserve
worthy opponents. Too often mathematics problems fall after a single
``trick.'' Some worthwhile problems take time to solve and cannot be
done in a single sitting.
\item To challenge the common view that mathematics is a body of knowledge 
to be memorized and repeated. The art and science of doing mathematics
is a process of reasoning and personal discovery followed by
justification and explanation. We wish to convey this to the reader,
and sincerely hope that the reader will pass this on to others as
well.
\end{itemize}
In summary---you, the reader, must become a doer of mathematics.  To
this end, many questions are asked in the text that follows. Sometimes
these questions are answered, other times the questions are left for
the reader to ponder. To let the reader know which questions are left
for cogitation, a large question mark is displayed:
\QM
The instructor of the course will address some of these questions. If
a question is not discussed to the reader's satisfaction, then I
encourage the reader to put on a thinking-cap and think, think, think!
If the question is still unresolved, go to the World Wide Web and
search, search, search!

This document is open-source. It is licensed under the Creative
Commons Attribution-NonCommercial-ShareAlike (CC BY-NC-SA)
License. Loosely speaking, this means that this document is available
for free. Anyone can get a free copy of this document (source or PDF)
from the following site:
\[
\texttt{http://www.math.osu.edu/\~{}snapp/1118/}
\]
Please report corrections, suggestions, gripes, complaints, and
criticisms to Bart Snapp at: \texttt{snapp@math.osu.edu}


\section*{Thanks and Acknowledgments}

A brief history of this document: In 2009, Greg Williams, a Master of
Arts in Teaching student at Coastal Carolina University, worked with
Bart Snapp to produce an early draft of the chapter on isometries. In
the summer and fall of 2011, Bart Snapp wrote the remaining chapters of this
set of notes. Jenny George made editorial changes and added activities
 in the fall of 2013.

Much thanks goes to Herb Clemens, Vic Ferdinand, Betsy McNeal, and
Daniel Shapiro who introduced me to ideas which have been used in this
course.



\tableofcontents













